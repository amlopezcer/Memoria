\chapter{Pruebas}

%Describir el escenario de pruebas usado (bicis de Madrid) y cómo lo has probado\dots Comentar el número de dispositivos, personas, herramientas (como el RESTClient de Firefox)

\section{Introducción}

Mediante la actividad conocida como \emph{aseguramiento de la calidad del software}\footnote{Para conocer más en detalle esta actividad se puede recurrir a~\cite{Pre10}.} se busca conseguir unos mínimos de calidad sobre las diversas etapas de ingeniería del software por las que pasa el producto: requerimientos, diseño e implementación (o código). Sin embargo, en este proceso pueden pasar errores inadvertidos que, sin unas pruebas adecuadamente planteadas y planificadas permanecerían en el sistema construido hasta su publicación.

Las pruebas representan, por tanto, la última oportunidad para valorar la calidad y, desde un punto de vista más práctico, descubrir errores. Sin embargo, citando a Pressman~\cite{Pre10}, ``no se puede probar la calidad. Si no está ahí antes de comenzar las pruebas, no estará cuando termine de probar''. Es decir, la calidad se ha de incorporar al software a lo largo de todo el proceso ingeniería, de modo que quede confirmada durante la etapa de pruebas.

Por tanto, el software se prueba para descubrir errores que se cometieron de manera inadvertida conforme se diseñó y construyó.

Las pruebas de software forman parte de un tema más amplio conocido como verificación y validación. La \emph{verificación} se refiere al conjunto de tareas que garantizan que el software implementa correctamente una función específica. La \emph{validación} es un conjunto diferente de tareas que aseguran que el software que se construye sigue los requerimientos establecidos en las etapas iniciales. La estrategia de pruebas se deberá enfocar a cubrir ambos aspectos: el sistema ha de funcionar adecuadamente de acuerdo a las funciones marcadas en la ERS (sección~\ref{sec:secERS}).

\section{Estrategia de pruebas}

Una estrategia de pruebas software proporciona una guía que incorpora la planificación de la prueba, el diseño de casos de prueba, la ejecución de la prueba y la recolección y evaluación de los resultados.
%seguir las notas del cuadreno...pero aquí hay que poner el entorno de pruebas: bicis de madrid, dispositivos (versiones android mío de Ana, de Jorge...), personas...mención a cositas de pruebas de unidad e integración pero por encima y citar directamnete al apartado siguiente de pruebas de validación.






\section{Pruebas de validación}

\subsection{Especificación de las pruebas planteadas}

\subsection{Resultados}