\chapter{Pruebas}

%Describir el escenario de pruebas usado (bicis de Madrid) y cómo lo has probado\dots Comentar el número de dispositivos, personas, herramientas (como el RESTClient de Firefox)

\section{Introducción}

Mediante la actividad conocida como \emph{aseguramiento de la calidad del software}\footnote{Para conocer más en detalle esta actividad se puede recurrir a~\cite{Pre10}.} se busca conseguir unos mínimos de calidad sobre las diversas etapas de ingeniería del software por las que pasa el producto: requerimientos, diseño e implementación (o código). Sin embargo, en este proceso pueden pasar errores inadvertidos que, sin unas pruebas adecuadamente planteadas y planificadas permanecerían en el sistema construido hasta su publicación.

Las pruebas representan, por tanto, la última oportunidad para valorar la calidad y, desde un punto de vista más práctico, descubrir errores. Sin embargo, citando a Pressman~\cite{Pre10}, ``no se puede probar la calidad. Si no está ahí antes de comenzar las pruebas, no estará cuando termine de probar''. Es decir, la calidad se ha de incorporar al software a lo largo de todo el proceso ingeniería, de modo que quede confirmada durante la etapa de pruebas.

Por tanto, el software se prueba para descubrir errores que se cometieron de manera inadvertida conforme se diseñó y construyó.

Las pruebas de software forman parte de un tema más amplio conocido como verificación y validación. La \emph{verificación} se refiere al conjunto de tareas que garantizan que el software implementa correctamente una función específica. La \emph{validación} es un conjunto diferente de tareas que aseguran que el software que se construye sigue los requerimientos establecidos en las etapas iniciales. La estrategia de pruebas se deberá enfocar a cubrir ambos aspectos: el sistema ha de funcionar adecuadamente de acuerdo a las funciones marcadas en la ERS (sección~\ref{sec:secERS}).

\section{Estrategia de pruebas}
%seguir las notas del cuadreno...pero aquí hay que poner el entorno de pruebas: bicis de madrid, dispositivos (versiones android mío de Ana, de Jorge...), personas...mención a cositas de pruebas de unidad e integración pero por encima y citar directamnete al apartado siguiente de pruebas de validación.

Una estrategia de pruebas software proporciona una guía que incorpora la planificación de la prueba, el diseño de casos de prueba, la ejecución de la prueba y la recolección y evaluación de los resultados. En esta sección, por tanto, se detallarán las pruebas realizadas, su entorno de control y los actores involucrados.

Dentro del contexto de la Ingeniería del Software, las pruebas suponen una serie de cuatro pasos que se implementan de manera secuencial, tal y como se muestra en la figura~\ref{fig:estrategiaPruebas}. 

\begin{figure}
	\centering
	\resizebox{9cm}{!}{\includegraphics[width=\linewidth,height=\textheight,keepaspectratio]{Images/estrategiaPruebas}}
	\caption{Pasos de las pruebas software \textit{Fuente:~\cite{Pre10}}}
	\label{fig:estrategiaPruebas}
\end{figure}

Inicialmente, las pruebas se enfocan en cada componente de manera individual, son las \emph{pruebas de unidad} y garantizan que todo módulo de software funciona adecuadamente como unidad, de ahí el nombre. A continuación, los componentes deben ensamblarse e integrarse para formar el paquete de software completo. La \emph{prueba de integración} aborda los conflictos asociados con los problemas de verificación y construcción de programas. Después de integrar (construir) el software, se realiza una serie de \emph{pruebas de orden superior}, donde se identifican las \emph{pruebas de validación}, que proporcionan la garantía final de que el software cumple con todos los requerimientos funcionales y de compartimento abordados en la fase de requisitos, y las \emph{pruebas del sistema}, para verificar que todos los elementos se mezclan de manera adecuada y que se logra el funcionamiento/rendimiento global del sistema deseado.

Considerando el proceso descrito, en los apartados siguientes se aportarán los elementos más destacables empleados para probar la aplicación construida; si bien en este documento sólo se indicará el detalle concreto de las \emph{pruebas de validación}, por ser aquellas que se centran en la funcionalidad específica del sistema y cuyo éxito depende del de las pruebas previas.

\subsection{Pruebas de unidad}

\subsection{Pruebas de integración}

\subsection{Pruebas de validación}

\subsubsection{Especificación de las pruebas planteadas}

\subsubsection{Resultados}

\subsection{Pruebas de sistema}

La \emph{prueba del sistema} es una serie de diferentes pruebas cuyo propósito principal es ejercitar por completo, saliendo de la Ingeniería del Software y entrando en la de Sistemas, el sistema construido. Algunos de los ejemplos más representativos son~\cite{Pre10}:

\begin{itemize}
	\item \textbf{Pruebas de recuperación}, dedicadas a la capacidad de relanzamiento (autónomo o con intervención humana) del software ante fallos que provoquen momentos de inactividad.
	\item \textbf{Pruebas de seguridad}, para asegurar el sistema ante ataques o intentos de entrada impropios.
	\item \textbf{Pruebas de esfuerzo}, destinadas a evaluar el comportamiento del sistema ante una demanda anormal de recursos.
	\item \textbf{Pruebas de rendimiento}, similares a las anteriores, estas pruebas están centradas en obtener unas métricas de funcionamiento mínimas.
	\item \textbf{Pruebas de despliegue o configuración}, para aquellos sistemas destinados a implementarse en varias plataformas, estas pruebas evalúan su desempeño en cada una de ellas.
\end{itemize}

Dado el tipo de pruebas encuadradas en este epígrafe, y considerando la naturaleza académica del proyecto aquí presentado, las \emph{pruebas de sistema} quedan fuera del alcance de la estrategia de pruebas para la aplicación construida. Algunas de las anteriores, sin embargo, sí se pueden asimilar a ciertas \emph{pruebas de validación} antes mencionadas, como la evaluación del correcto funcionamiento en diferentes plataformas como \emph{prueba de despliegue}, o el impedimento de acceso a un usuario no registrado como \emph{prueba de seguridad}.



