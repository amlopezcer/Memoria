\chapter{Objetivos}
\label{ch:objetivos}

El objetivo primario del presente proyecto ha sido el desarrollo de una aplicación móvil para la gestión de un parque público de bicicletas. De manera desglosada:

\begin{itemize}  
	\item Desarrollar una aplicación Android comunicada con un servidor y base de datos mediante servicios web que implemente un sistema de gestión de bicicletas para coger, dejar o reservar bicicletas y anclajes a usuarios registrados.
	\item Realizar un adecuado entendimiento de las tecnologías disponibles para la implementación de la infraestructura mencionada, como podrían ser protocolos y estándares de comunicación, servidores, sistemas gestores de bases de datos, etc.
	\item Introducir un primer nivel de adaptación dinámica de precios dependiendo de la disponibilidad individual de cada estación de bicicletas.
	\item Evitar condiciones de carrera ocasionadas como consecuencia de accesos simultáneos a un mismo recurso.
	\item Seguir las recomendaciones generales de diseño e implementación para asegurar una aplicación fácilmente mantenible.
	\item Seguir las normas y recomendaciones generales y específicas de las plataformas utilizadas para el diseño de interfaces de usuario, de modo que se garantice una adecuada experiencia de usuario.
	\item Cubrir el mayor volumen de dispositivos posible en diferentes ámbitos como el tecnológico (versiones Android), lingüístico (diferentes idiomas), etc.
	\item Desarrollar una adecuada estrategia de pruebas que asegure una aplicación consistente y libre de errores en sus principales funcionalidades.
	\item Realizar un estudio conveniente, mediante bibliografía especializada, para la adecuada descripción y documentación del sistema.
	\item Hacer uso de un sistema controlador de versiones (CVS) que permita tener una traza adecuada del progreso y deje abierta la posibilidad de volver a versiones previas en caso de necesidad.
\end{itemize}


