\chapter{Introducción}

\section{Motivación}

Los parques públicos de bicicletas están cada vez más extendidos en las ciudades como alternativa a los medios de transporte tradicionales. Asimismo, los dispositivos móviles representan, para gran parte de la población, la herramienta básica para el manejo de numerosas tareas diarias.

Ambos factores suponen la motivación básica para el desarrollo de una aplicación de gestión de dichos parques, de manera que los usuarios registrados puedan conocer y operar con las estaciones disponibles y se logre una experiencia de usuario positiva al mismo tiempo que se logre un modelo de gestión eficiente.

El presente proyecto surge con la idea de suponer una infraestructura tecnológica base para el desarrollo e investigaciones futuras sobre el área que mejoren la gestión del conjunto de estaciones.


\section{Objetivos}
\label{sec:objetivos}

El objetivo primario del presente proyecto ha sido el desarrollo de una aplicación móvil distribuida para la gestión de un parque público de bicicletas. De manera desglosada:

\begin{itemize}  
	\item Desarrollar una aplicación Android comunicada con un servidor y base de datos mediante servicios web que implemente un sistema de gestión de bicicletas para coger, dejar o reservar bicicletas y anclajes a usuarios registrados.
	\item Realizar un adecuado entendimiento de las tecnologías disponibles para la implementación de la infraestructura mencionada, como podrían ser protocolos y estándares de comunicación, servidores, sistemas gestores de bases de datos, etc.
	\item Introducir un primer nivel de adaptación dinámica de precios dependiendo de la disponibilidad individual de cada estación de bicicletas.
	\item Evitar condiciones de carrera ocasionadas como consecuencia de accesos simultáneos a un mismo recurso.
	\item Seguir las recomendaciones generales de diseño e implementación para asegurar una aplicación fácilmente mantenible.
	\item Seguir las normas y recomendaciones generales y específicas de las plataformas utilizadas para el diseño de interfaces de usuario, de modo que se garantice una adecuada experiencia de usuario.
	\item Cubrir el mayor volumen de dispositivos posible en diferentes ámbitos como el tecnológico (versiones Android), lingüístico (diferentes idiomas), etc.
	\item Desarrollar una adecuada estrategia de pruebas que asegure una aplicación consistente y libre de errores en sus principales funcionalidades.
	\item Realizar un estudio conveniente, mediante bibliografía especializada, para la adecuada descripción y documentación del sistema.
	\item Hacer uso de un sistema controlador de versiones (CVS) que permita tener una traza adecuada del progreso y deje abierta la posibilidad de volver a versiones previas en caso de necesidad.
\end{itemize}


\section{Metodología}

El modelo de proceso seguido ha sido el \emph{modelo incremental}, que combina elementos de los flujos de proceso lineal y paralelo. Como se puede observar en la figura~\ref{fig:procesoIncremental}, el modelo aplica secuencias lineales en forma escalonada a medida que avanza el calendario de actividades. Cada secuencia lineal produce \emph{incrementos} de software susceptibles de entregarse de manera parecida a los incrementos producidos en un flujo de proceso evolutivo.

\begin{figure}
	\centering
	\includegraphics[width=\linewidth,height=\textheight,keepaspectratio]{Images/ModeloIncremental}
	\caption[El modelo incremental]{El modelo incremental. \textit{Fuente:~\cite{Pre10}}}
	\label{fig:procesoIncremental}
\end{figure}

Cuando se utiliza un modelo incremental, es frecuente que el primer incremento sea el \emph{producto fundamental}. Es decir, se abordan los requerimientos básicos, pero no se proporcionan muchas características suplementarias (algunas conocidas y otras no). Como resultado del uso y/o evaluación de este primer producto, se desarrolla un plan para el incremento que sigue. El plan incluye la modificación del producto fundamental para cumplir mejor las nuevas necesidades, así como la entrega de características adicionales y más funcionalidad. Este proceso se repite después de entregar cada incremento, hasta terminar el producto final.

El modelo de proceso incremental se centra en que en cada incremento se entrega un producto que ya opera. Los primeros incrementos son versiones desnudas del producto final, pero proporcionan capacidad que sirve al usuario y también le dan una plataforma de evaluación \cite{Pre10}.

Considerando las características anteriores, los incrementos generales establecidos han sido los siguientes:

\begin{enumerate}  
	\item Mapa con estaciones sobre las que poder operar con datos locales.
	\item Desarrollo e integración con la aplicación del Servidor y base de datos.
	\item Sistema gestor de usuarios.
	\item Sistema de gestión de reservas ligadas al usuario.
	\item Listado con el detalle de las estaciones mostradas en el mapa.
	\item Gráficos con el estado del parque de estaciones.
\end{enumerate}


