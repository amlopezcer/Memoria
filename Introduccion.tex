\chapter{Introducción}

\section{Motivación}

Los parques públicos de bicicletas están cada vez más extendidos en las ciudades como alternativa a los medios de transporte tradicionales. Asimismo, los dispositivos móviles representan, para gran parte de la población, la herramienta básica para el manejo de numerosas tareas diarias.

Ambos factores suponen la motivación básica para el desarrollo de una aplicación de gestión de dichos parques, de manera que los usuarios registrados puedan conocer y operar con las estaciones disponibles y se logre una experiencia de usuario positiva al mismo tiempo que se realice una gestión eficiente del parque.

El presente proyecto surge con la idea de suponer una infraestructura tecnológica base para el desarrollo e investigaciones futuras sobre el área que mejoren la gestión del conjunto de estaciones.

\section{Metodología}

El modelo de proceso seguido ha sido el \emph{modelo incremental}, que combina elementos de los flujos de proceso lineal y paralelo. Como se puede observar en la figura ~\ref{fig:procesoIncremental}, el modelo aplica secuencias lineales en forma escalonada a medida que avanza el calendario de actividades. Cada secuencia lineal produce \emph{incrementos} de software susceptibles de entregarse de manera parecida a los incrementos producidos en un flujo de proceso evolutivo.

\begin{figure}
	\centering
	\includegraphics[width=\linewidth,height=\textheight,keepaspectratio]{Images/ModeloIncremental}
	\caption{El modelo incremental, de Pressman 2010}
	\label{fig:procesoIncremental}
\end{figure}

Cuando se utiliza un modelo incremental, es frecuente que el primer incremento sea el \emph{producto fundamental}. Es decir, se abordan los requerimientos básicos, pero no se proporcionan muchas características suplementarias (algunas conocidas y otras no). Como resultado del uso y/o evaluación de este primer producto, se desarrolla un plan para el incremento que sigue. El plan incluye la modificación del producto fundamental para cumplir mejor las nuevas necesidades, así como la entrega de características adicionales y más funcionalidad. Este proceso se repite después de entregar cada incremento, hasta terminar el producto final.

El modelo de proceso incremental se centra en que en cada incremento se entrega un producto que ya opera. Los primeros incrementos son versiones desnudas del producto final, pero proporcionan capacidad que sirve al usuario y también le dan una plataforma de evaluación \cite{Pre10}.

Considerando las características anteriores, los incrementos generales establecidos han sido los siguientes:

\begin{enumerate}  
	\item Mapa con estaciones sobre las que poder operar con datos locales.
	\item Lista con el detalle de las estaciones mostradas en el mapa.
	\item Gráficos con el estado del parque de estaciones.
	\item Desarrollo e integración con la aplicación del Servidor y base de datos.
	\item Sistema gestor de usuarios.
	\item Sistema de gestión de reservas ligadas al usuario.
\end{enumerate}

