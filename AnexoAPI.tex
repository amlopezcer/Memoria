\chapter{API implementada}
\label{app:appAPI}

En este anexo se aporta el detalle de la API implementada en el servidor de aplicación tal y como se introdujo en la sección~\ref{subsec:disenoInterfaces}, referida al diseño de interfaces.

Para una mejor comprensión, la definición de la API en el presente documento se ha dividido en dos niveles: un primer nivel introductorio en el que, por método, se especifica su ruta y objetivo concreto; y un segundo nivel más detallado en el que se especifican los parámetros de entrada, salida y ejemplos asociados. Señalar que todas las salidas contempladas se describen suponiendo un mensaje de respuesta HTTP 200 (ejecución correcta del método).

Cabe recordar que las URLs se construyen a partir de la plantilla $$http://[IP\_SERV]:[PORT]/BikesManager/rest/entities.[ENTITY]$$ (ver sección referenciada anteriormente para más detalles), sobre la que se construyen las llamadas particulares a los métodos mediante parámetros adicionales sobre la URL, tal y como queda descrito a continuación.

\subsubsection{Introducción a la API}

\newcolumntype{A}[1]{>{\arraybackslash}m{#1}} %Sin la definción de una columna propia (A), la sección de comentarios no se ajusta bien a la página
\newcolumntype{B}[1]{>{\centering\arraybackslash}m{#1}} 

\begin{itemize}
	\item \textbf{Estación}
	
	\begin{center}
		\begin{longtable}{|c|c|B{2.5cm}|A{8cm}|}
			\hline
			
			\multicolumn{1}{|c|}{\textbf{ID}}	& \textbf{Método}	& \textbf{Ruta}	& \textbf{Objetivo} \\ 	\hline
			\endfirsthead
			
			\multicolumn{4}{c}{{\bfseries \tablename\ \thetable{} -- Continúa desde la página anterior}} 			\\	\hline
			\multicolumn{1}{|c|}{\textbf{ID}}	& \textbf{Método}	& \textbf{Ruta}	& \textbf{Objetivo} \\ 	\hline
			\endhead
			
			\hline \multicolumn{4}{|r|}{{\textit{Continúa en la página siguiente}}} \\ \hline
			\endfoot
			
			\endlastfoot
			
			\multirow{1}{*}{1}	& POST	& -	& Método básico para la creación de entidades de tipo Estación en base de datos. Este método, si bien está disponible, no se ha utilizado en la aplicación desarrollada puesto que las estaciones se crean directamente en base de datos. \\ \hline
			
			\multirow{1}{*}{2}	& PUT	& \{id\}	& Método básico para la actualización directa (sin comprobaciones) de entidades de tipo Estación en la base de datos. \\ \hline
			
			\multirow{1}{*}{3}	& PUT	& \{operation\} / \{id\}	& Método para la actualización de Estaciones cuando se realiza alguna de las siguientes operaciones: coger bicicleta, dejar bicicleta, reservar bicicleta o anclaje. Este método se ha implementado adicionalmente al anterior puesto que las operaciones mencionadas requieren de comprobaciones sobre el servidor para evaluar la disponibilidad de cada estación y ofrecer al usuario un resultado consistente. \\ \hline
			
			\multirow{1}{*}{4}	& DELETE	& \{id\}	& Método básico para el borrado de entidades de tipo Estación de la base de datos. Este método no se ha utilizado en la aplicación desarrollada puesto que las estaciones se eliminan directamente en base de datos. \\ \hline
			
			\multirow{1}{*}{5}	& GET	& \{id\}	& Método básico para obtener entidades de tipo Estación de la base de datos a partir de su ID.	\\ \hline
			
			\multirow{1}{*}{6}	& GET	& sta\-tion\-Ad\-dress / \{ad\-dress\}	& Método adicional al anterior para obtener entidades de tipo Estación de la base de datos a partir de su dirección (campo único). \\ \hline
			
			\multirow{1}{*}{7}	& GET	& -	& Método básico para la obtención del listado completo de entidades de tipo Estación creadas en la base de datos. \\ \hline
			
			\multirow{1}{*}{8}	& GET	& \{from\} / \{to\}	& Método básico para la obtención de un listado acotado de entidades de tipo Estación creadas en la base de datos. Se obtienen tantas entidades como las indicadas en el parámetro \textit{to} a partir de la estación con el ID más cercano al indicado en el parámetro \textit{from}. \\ \hline
			
			\multirow{1}{*}{9}	& GET	& count	& Método básico para conocer el número de entidades de tipo Estación creadas en la base de datos. \\ \hline
			
			\caption{Vista general de la API para la entidad Estación}
			\label{tab:apiEstacionGeneral}
		\end{longtable}
	\end{center}
	
	\item \textbf{Usuario}
	
	\begin{center}
		\begin{longtable}{|c|c|B{2.5cm}|A{8cm}|}
			\hline
			
			\multicolumn{1}{|c|}{\textbf{ID}}	& \textbf{Método}	& \textbf{Ruta}	& \textbf{Objetivo} \\ 	\hline
			\endfirsthead
			
			\multicolumn{4}{c}{{\bfseries \tablename\ \thetable{} -- Continúa desde la página anterior}} 			\\	\hline
			\multicolumn{1}{|c|}{\textbf{ID}}	& \textbf{Método}	& \textbf{Ruta}	& \textbf{Objetivo} \\ 	\hline
			\endhead
			
			\hline \multicolumn{4}{|r|}{{\textit{Continúa en la página siguiente}}} \\ \hline
			\endfoot
			
			\endlastfoot
			
			\multirow{1}{*}{1}	& POST	& -	& Método básico para la creación de entidades de tipo Usuario en la base de datos. Este método comprueba la disponibilidad del nombre de usuario o dirección de correo elegidas, informando al usuario de si la creación ha sido posible o no.	\\ \hline
			
			\multirow{1}{*}{2}	& PUT	& \{id\}	& Método básico para la actualización directa (sin comprobaciones) de entidades de tipo Usuario en la base de datos. \\ \hline
			
			\multirow{1}{*}{3}	& PUT	& ba\-sic\-da\-ta / \{id\}	& Método adicional al anterior para actualizar Usuarios cuando se quiere modificar el nombre de usuario o dirección de correo, modificación que requieren de comprobaciones de disponibilidad adicionales. \\ \hline
			
			\multirow{1}{*}{4}	& DELETE	& \{id\}	& Método básico para el borrado de entidades de tipo Usuario de la base de datos. \\ \hline
			
			\multirow{1}{*}{5}	& GET	& \{id\}	& Método básico para obtener usuarios de la base de datos a partir de su ID. \\ \hline
			
			\multirow{1}{*}{6}	& GET	& u\-ser / \{u\-ser\-na\-me\}	& Método adicional al anterior para obtener usuarios de la base datos a partir del nombre de usuario.	\\ \hline
			
			\multirow{1}{*}{7}	& GET	& -	&  Método básico para la obtención del listado completo de entidades de tipo Usuario creadas en la base de datos.	\\ \hline
			
			\multirow{1}{*}{8}	& GET	& \{from\} / \{to\}	& Método básico para la obtención de un listado acotado de entidades de tipo Usuario creadas en la base de datos. Se obtienen tantas entidades como las indicadas en el parámetro \textit{to} a partir del usuario con el ID más cercano al indicado en el parámetro \textit{from}. \\ \hline
			
			
			\multirow{1}{*}{9}	& GET	& count	& Método básico para conocer el número de entidades de tipo Usuario creadas en la base de datos.	\\ \hline
			
			\caption{Vista general de la API para la entidad Usuario}
			\label{tab:apiUsuarioGeneral}
		\end{longtable}
	\end{center}
	
	\item \textbf{Reserva}
	
	\begin{center}
		\begin{longtable}{|c|c|B{2.5cm}|A{8cm}|}
			\hline
			
			\multicolumn{1}{|c|}{\textbf{ID}}	& \textbf{Método}	& \textbf{Ruta}	& \textbf{Objetivo} \\ 	\hline
			\endfirsthead
			
			\multicolumn{4}{c}{{\bfseries \tablename\ \thetable{} -- Continúa desde la página anterior}} 			\\	\hline
			\multicolumn{1}{|c|}{\textbf{ID}}	& \textbf{Método}	& \textbf{Ruta}	& \textbf{Objetivo} \\ 	\hline
			\endhead
			
			\hline \multicolumn{4}{|r|}{{\textit{Continúa en la página siguiente}}} \\ \hline
			\endfoot
			
			\endlastfoot
			
			\multirow{1}{*}{1}	& POST	& -	& Método básico para la creación de entidades de tipo Reserva en la base de datos.	\\ \hline
			
			\multirow{1}{*}{2}	& PUT	& \{id\}	& Método básico para la actualización directa (sin comprobaciones) de entidades de tipo Reserva en la base de datos. Este método, si bien queda disponible, no se ha utilizado en la aplicación desarrollada puesto que las reservas sólo se pueden crear o cancelar (eliminar). \\ \hline
			
			\multirow{1}{*}{3}	& DELETE	& \{id\}	& Método básico para el borrado de entidades de tipo Reserva de la base de datos. Este método, si bien queda disponible, no se ha utilizado, puesto que para el borrado de reservas se ha implementado el método descrito a continuación. \\ \hline
			
			\multirow{1}{*}{4}	& DELETE	& \{{u\-ser\-na\-me}\} / \{{book\-ing\-ty\-pe}\}	&	Método adicional al anterior para el borrado de entidades de tipo Reserva de la base de datos. Puesto que un usuario sólo puede cancelar sus propias reservas, este método realiza, bajo petición, una búsqueda del tipo de reserva especificado a su nombre y la elimina.	\\ \hline
			
			\multirow{1}{*}{5}	& GET	& \{id\}	& Método básico para obtener reservas de la base de datos a partir de su ID.	\\ \hline
			
			\multirow{1}{*}{6}	& GET	& -	& Método básico para la obtención del listado completo de entidades de tipo Reserva creadas en la base de datos.  \\ \hline
			
			\multirow{1}{*}{7}	& GET	& \{from\} / \{to\}	& Método básico para la obtención de un listado acotado de entidades de tipo Reserva creadas en la base de datos. Se obtienen tantas entidades como las indicadas en el parámetro \textit{to} a partir de la reserva con el ID más cercano al indicado en el parámetro \textit{from}. \\ \hline
			
			\multirow{1}{*}{8}	& GET	& count	& Método básico para conocer el número de entidades de tipo Reserva creadas en la base de datos. \\ \hline
			
			\caption{Vista general de la API para la entidad Reserva}
			\label{tab:apiReservaGeneral}
		\end{longtable}
	\end{center}
	
\end{itemize}


\subsubsection{Detalle de la API}

%\newcolumntype{A}[1]{>{\arraybackslash}m{#1}}
%\newcolumntype{B}[1]{>{\centering\arraybackslash}m{#1}} 
\begin{itemize}
	\item \textbf{Estación}
	
	\begin{center}
		{\tiny
			\begin{longtable}{|c|c|B{1.2cm}|A{2cm}|A{2cm}|A{2cm}|A{2.7cm}|}
				\hline
				
				\multicolumn{1}{|c|}{\textbf{ID}}	& \textbf{Método}	& \textbf{Ruta}	& \textbf{Entrada} & \textbf{Salida}	& \textbf{Ejemplo} & \textbf{Comentarios} \\ 	\hline
				\endfirsthead
				
				\multicolumn{7}{c}{{\bfseries \tablename\ \thetable{} -- Continúa desde la página anterior}} 			\\	\hline
				\multicolumn{1}{|c|}{\textbf{ID}}	& \textbf{Método}	& \textbf{Ruta}	& \textbf{Entrada} & \textbf{Salida}	& \textbf{Ejemplo} & \textbf{Comentarios} \\ 	\hline
				\endhead
				
				\hline \multicolumn{7}{|r|}{{\textit{Continúa en la página siguiente}}} \\ \hline
				\endfoot
				
				\endlastfoot
				
				\multirow{1}{*}{1}	& POST	& -	& -	& -	& \url{http://localhost:8080/BikesManager/rest/entities.bikestation}	& La petición ha de venir acompañada de una cadena JSON con la entidad a crear, si bien este método no se ha utilizado en la aplicación desarrollada puesto que las estaciones se crean directamente en base de datos.	\\ \hline
				
				\multirow{1}{*}{2}	& PUT	& \{id\}	& \textbf{id}: \emph{Integer} con el ID de la estación a actualizar. & JSON: mensaje \emph{en\-ti\-ty\-\_bi\-ke\-sta\-tion\-\_SER\-VER\-\_OK}. & \url{http://localhost:8080/BikesManager/rest/entities.bikestation/1}	& La petición ha de venir acompañada de una cadena JSON con la entidad a actualizar. Este método supone la actualización básica de la entidad, sin necesidad de realizar comprobaciones previas para ello, con lo que siempre devolverá SERVER\_OK una vez ejecutada. \\ \hline
				
				\multirow{1}{*}{3}	& PUT	& \{operation\} / \{id\}	& \textbf{operation}: \emph{String} con la operación a realizar sobre la estación (\textit{take}, \textit{leave}, \textit{book\_bike} o \textit{book\_slot}); \textbf{id}: \emph{Integer} con el ID de la estación a actualizar. & JSON: mensaje \emph{en\-ti\-ty\-\_bi\-ke\-sta\-tion\-\_SER\-VER\-\_OK} o \emph{en\-ti\-ty\-\_bi\-ke\-sta\-tion\-\_SER\-VER\-\_KO}.  & \url{http://localhost:8080/BikesManager/rest/entities.bikestation/take/1}	& La petición ha de venir acompañada de una cadena JSON con la entidad a actualizar. Este método es el utilizado para actualizar la estación cuando se cogen o dejan bicicletas o se reservan bicicletas o anclajes, de modo que se devuelve SERVER\_OK si la operación ha sido posible basándose en condiciones de disponibilidad de la estación y SERVER\_KO en caso contrario. \\ \hline
				
				\multirow{1}{*}{4}	& DELETE	& \{id\}	& \textbf{id}: \emph{Integer} con el ID de la estación a eliminar.  & -	& \url{http://localhost:8080/BikesManager/rest/entities.bikestation/1}	& Este método no se ha utilizado en la aplicación desarrollada puesto que las estaciones se eliminan directamente en base de datos. \\ \hline
				
				\multirow{1}{*}{5}	& GET	& \{id\}	& \textbf{id}: \emph{Integer} con el ID de la estación a buscar.	& JSON: estación buscada.	& \url{http://localhost:8080/BikesManager/rest/entities.bikestation/1}	&  Este método no se ha utilizado en la aplicación desarrollada, puesto que las búsquedas individuales se realizan sobre la dirección, no sobre el ID. \\ \hline
				
				\multirow{1}{*}{6}	& GET	& sta\-tion\-Ad\-dress / \{ad\-dress\}	& \textbf{address}: \emph{String} con la dirección de la estación a buscar.  & JSON: estación buscada.	& \url{http://localhost:8080/BikesManager/rest/entities.bikestation/stationAddress/Plaza_de_la_Puerta_del_Sol}	& Los elementos de la dirección han de venir separados por ``\_'' y no por espacios en blanco en la URL. \\ \hline
				
				\multirow{1}{*}{7}	& GET	& -	& -	& JSON: listado completo de estaciones.	& \url{http://localhost:8080/BikesManager/rest/entities.bikestation}	& Este método, antes de devolver el listado de estaciones completo, actualiza las posibles reservas caducadas. Es el utilizado cuando se actualiza el estado global del sistema.  \\ \hline
				
				\multirow{1}{*}{8}	& GET	& \{from\} / \{to\}	& \textbf{from}: \emph{Integer} con el ID de la estación desde la que se parte para la búsqueda; \textbf{to}: \emph{Integer} con el número de estaciones a obtener.  & JSON: rango de estaciones buscado.	& \url{http://localhost:8080/BikesManager/rest/entities.bikestation/1/5}	& Este método no se ha utilizado en la aplicación desarrollada. \\ \hline
				
				
				\multirow{1}{*}{9}	& GET	& count	& -	& PLAIN: número de entidades creadas en la base de datos.	& \url{http://localhost:8080/BikesManager/rest/entities.bikestation/count}	& Este método no se ha utilizado en la aplicación desarrollada. \\ \hline
				
				\caption{API para la entidad Estación}
				\label{tab:apiEstacion}
			\end{longtable}
		}
	\end{center}
	
	\newpage
	\item \textbf{Usuario}
	
	\begin{center}
		{\tiny
			\begin{longtable}{|c|c|B{1.2cm}|A{2cm}|A{2cm}|A{2cm}|A{2.7cm}|}
				\hline
				
				\multicolumn{1}{|c|}{\textbf{ID}}	& \textbf{Método}	& \textbf{Ruta}	& \textbf{Entrada} & \textbf{Salida}	& \textbf{Ejemplo} & \textbf{Comentarios} \\ 	\hline
				\endfirsthead
				
				\multicolumn{7}{c}{{\bfseries \tablename\ \thetable{} -- Continúa desde la página anterior}} 			\\	\hline
				\multicolumn{1}{|c|}{\textbf{ID}}	& \textbf{Método}	& \textbf{Ruta}	& \textbf{Entrada} & \textbf{Salida}	& \textbf{Ejemplo} & \textbf{Comentarios} \\ 	\hline
				\endhead
				
				\hline \multicolumn{7}{|r|}{{\textit{Continúa en la página siguiente}}} \\ \hline
				\endfoot
				
				\endlastfoot
				
				\multirow{1}{*}{1}	& POST	& -	& -	& JSON: mensaje \emph{en\-ti\-ty\-\_bi\-ke\-u\-ser\-\_SER\-VER\-\_OK} o \emph{en\-ti\-ty\-\_bi\-ke\-u\-ser\-\_SER\-VER\-\_KO}.	& \url{http://localhost:8080/BikesManager/rest/entities.bikeuser}	& La petición ha de venir acompañada de una cadena JSON con la entidad a crear. Este método comprueba la disponibilidad del nombre de usuario y correo electrónico, de modo que devuelve un SERVER\_OK (crea el usuario) si no hay coincidencias o un SERVER\_KO (no se crea el usuario) en caso contrario.	\\ \hline
				
				\multirow{1}{*}{2}	& PUT	& \{id\}	& \textbf{id}: \emph{Integer} con el ID del usuario a actualizar. & JSON: mensaje \emph{en\-ti\-ty\-\_bi\-ke\-u\-ser\-\_SER\-VER\-\_OK}. & \url{http://localhost:8080/BikesManager/rest/entities.bikeuser/1}	& La petición ha de venir acompañada de una cadena JSON con la entidad a actualizar. Este método supone la actualización básica de la entidad, sin necesidad de realizar comprobaciones previas para ello, con lo que siempre devolverá SERVER\_OK una vez ejecutada. \\ \hline
				
				\multirow{1}{*}{3}	& PUT	& ba\-sic\-da\-ta / \{id\}	& \textbf{id}: \emph{Integer} con el ID del usuario a actualizar. & JSON: mensaje \emph{en\-ti\-ty\-\_bi\-ke\-u\-ser\-\_SER\-VER\-\_OK} o \emph{en\-ti\-ty\-\_bi\-ke\-u\-ser\-\_SER\-VER\-\_KO}.  & \url{http://localhost:8080/BikesManager/rest/entities.bikeuser/basicdata/1}	& La petición ha de venir acompañada de una cadena JSON con la entidad a actualizar. Este método es el utilizado para actualizar el usuario cuando se quieren modificar el nombre de usuario o la dirección de correo electrónico, devolviendo SERVER\_OK o SERVER\_KO dependiendo de la disponibilidad de los nuevos datos introducidos. \\ \hline
				
				\multirow{1}{*}{4}	& DELETE	& \{id\}	& \textbf{id}: \emph{Integer} con el ID del usuario a eliminar.  & JSON: mensaje \emph{en\-ti\-ty\-\_bi\-ke\-u\-ser\-\_SER\-VER\-\_OK}.	& \url{http://localhost:8080/BikesManager/rest/entities.bikeuser/1}	& Este método es el utilizado para eliminar usuarios. Como la petición siempre parte del propio usuario que quiere eliminar su cuenta, no hay posibilidad de error, devolviendo siempre un SERVER\_OK. \\ \hline
				
				\multirow{1}{*}{5}	& GET	& \{id\}	& \textbf{id}: \emph{Integer} con el ID del usuario a buscar.	& JSON: usuario buscado.	& \url{http://localhost:8080/BikesManager/rest/entities.bikeuser/1}	&  Este método no se ha utilizado en la aplicación desarrollada, puesto que las búsquedas individuales se realizan sobre el nombre de usuario, no sobre el ID. \\ \hline
				
				\multirow{1}{*}{6}	& GET	& u\-ser / \{u\-ser\-na\-me\}	& \textbf{username}: \emph{String} con el nombre de usuario a buscar.  & JSON: usuario buscado.	& \url{http://localhost:8080/BikesManager/rest/entities.bikeuser/user/usuario1}	& -	\\ \hline
				
				\multirow{1}{*}{7}	& GET	& -	& -	& JSON: listado completo de usuarios.	& \url{http://localhost:8080/BikesManager/rest/entities.bikeuser}	& Método no utilizado puesto que sólo se realizan consultas de usuarios individuales desde la aplicación.  \\ \hline
				
				\multirow{1}{*}{8}	& GET	& \{from\} / \{to\}	& \textbf{from}: \emph{Integer} con el ID del usuario desde el que se parte para la búsqueda; \textbf{to}: \emph{Integer} con el número de usuarios a obtener.  & JSON: rango de usuarios buscado.	& \url{http://localhost:8080/BikesManager/rest/entities.bikeuser/1/5}	& Este método no se ha utilizado en la aplicación desarrollada. \\ \hline
				
				
				\multirow{1}{*}{9}	& GET	& count	& -	& PLAIN: número de entidades creadas en la base de datos.	& \url{http://localhost:8080/BikesManager/rest/entities.bikeuser/count}	& Este método no se ha utilizado en la aplicación desarrollada. \\ \hline
				
				\caption{API para la entidad Usuario}
				\label{tab:apiUsuario}
			\end{longtable}
		}
	\end{center}
	
	\item \textbf{Reserva}
	
	\begin{center}
		{\tiny
			\begin{longtable}{|c|c|B{1.2cm}|A{2cm}|A{2cm}|A{2cm}|A{2.7cm}|}
				\hline
				
				\multicolumn{1}{|c|}{\textbf{ID}}	& \textbf{Método}	& \textbf{Ruta}	& \textbf{Entrada} & \textbf{Salida}	& \textbf{Ejemplo} & \textbf{Comentarios} \\	\hline
				\endfirsthead
				
				\multicolumn{7}{c}{{\bfseries \tablename\ \thetable{} -- Continúa desde la página anterior}}	\\	\hline
				\multicolumn{1}{|c|}{\textbf{ID}}	& \textbf{Método}	& \textbf{Ruta}	& \textbf{Entrada} & \textbf{Salida}	& \textbf{Ejemplo} & \textbf{Comentarios} \\ 	\hline
				\endhead
				
				\hline \multicolumn{7}{|r|}{{\textit{Continúa en la página siguiente}}} \\ \hline
				\endfoot
				
				\endlastfoot
				
				\multirow{1}{*}{1}	& POST	& -	& -	& JSON: mensaje \emph{en\-ti\-ty\-\_book\-ing\-\_SER\-VER\-\_OK}.	& \url{http://localhost:8080/BikesManager/rest/entities.booking}	& La petición ha de venir acompañada de una cadena JSON con la entidad a crear. Este método se ejecuta únicamente cuando la reserva puede ser verdaderamente creada por condiciones de disponibilidad, con lo que siempre devuelve SERVER\_OK.	\\ \hline
				
				\multirow{1}{*}{2}	& PUT	& \{id\}	& \textbf{id}: \emph{Integer} con el ID de la reserva a actualizar. & -	& \url{http://localhost:8080/BikesManager/rest/entities.booking/1}	& La petición ha de venir acompañada de una cadena JSON con la entidad a actualizar, si bien este método no se ha utilizado puesto que las reservas sólo se crean o borran, en ningún caso se actualizan. \\ \hline
				
				\multirow{1}{*}{3}	& DELETE	& \{id\}	& \textbf{id}: \emph{Integer} con el ID de la reserva a eliminar.  & -	& \url{http://localhost:8080/BikesManager/rest/entities.booking/1}	& Este método no se ha utilizado en la aplicación desarrollada, las reservas se eliminan mediante una búsqueda por el nombre de usuario que la realizó, no por el ID de la misma. \\ \hline
				
				\multirow{1}{*}{4}	& DELETE	& \{{u\-ser\-na\-me}\} / \{{book\-ing\-ty\-pe}\}	& \textbf{{u\-ser\-na\-me}}: \emph{String} con el nombre de usuario que solicita la eliminación de la reserva; \textbf{{book\-ing\-ty\-pe}}: \emph{Integer} con el tipo de reserva a cancelar (1 para las bicis y 2 para los anclajes).  & JSON: mensaje \emph{en\-ti\-ty\-\_book\-ing\-\_SER\-VER\-\_OK}.	& \url{http://localhost:8080/BikesManager/rest/entities.booking/usuario1/1}	&  Este método lo ejecuta el usuario sobre sus propias reservas, no hay posibilidad de error, devolviendo siempre SERVER\_OK.	\\ \hline
				
				\multirow{1}{*}{5}	& GET	& \{id\}	& \textbf{id}: \emph{Integer} con el ID de la reserva a buscar.	& JSON: reserva buscado.	& \url{http://localhost:8080/BikesManager/rest/entities.booking/1}	&  Este método no se ha utilizado en la aplicación desarrollada. \\ \hline
				
				\multirow{1}{*}{6}	& GET	& -	& -	& JSON: listado completo de reservas.	& \url{http://localhost:8080/BikesManager/rest/entities.booking}	& Este método no se ha utilizado en la aplicación desarrollada.  \\ \hline
				
				\multirow{1}{*}{7}	& GET	& \{from\} / \{to\}	& \textbf{from}: \emph{Integer} con el ID de la reserva desde la que se parte para la búsqueda; \textbf{to}: \emph{Integer} con el número de reservas a obtener.  & JSON: rango de reservas buscado.	& \url{http://localhost:8080/BikesManager/rest/entities.booking/1/5}	& Este método no se ha utilizado en la aplicación desarrollada. \\ \hline
				
				\multirow{1}{*}{8}	& GET	& count	& -	& PLAIN: número de entidades creadas en la base de datos.	& \url{http://localhost:8080/BikesManager/rest/entities.booking/count}	& Este método no se ha utilizado en la aplicación desarrollada. \\ \hline
				
				\caption{API para la entidad Reserva}
				\label{tab:apiReserva}
			\end{longtable}
		}
	\end{center}
	
\end{itemize}

Para conocer el formato de las cadenas JSON a introducir o a obtener, basta con ejecutar cualquiera de los métodos GET directamente sobre un navegador.