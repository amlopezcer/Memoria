\chapter{Manual de usuario}
\label{app:appManualUsuario}

\subsubsection{Configuración inicial y pantalla inicial}

Si se acaba de instalar la aplicación, se solicitará automáticamente la dirección del servidor para poder conectarse (esta información puede modificarse posteriormente):

\begin{figure} [!htb]
	\centering
	\resizebox{12cm}{!}{\includegraphics[width=\linewidth,height=\textheight,keepaspectratio]{Images/AnexoManual/01_inicio_01_conexion}}
	\caption{Configuración inicial de la conexión}
	\label{fig:confIniCon}
\end{figure}

A partir de este punto, se tiene acceso a la pantalla inicial de la aplicación, desde la que se podrá registrar un nuevo usuario o iniciar sesión con uno existente.

\begin{figure} [!htb]
	\centering
	\resizebox{5cm}{!}{\includegraphics[width=\linewidth,height=\textheight,keepaspectratio]{Images/AnexoManual/01_inicio_02_completa}}
	\caption{Pantalla inicial de la aplicación}
	\label{fig:pantallaInicial}
\end{figure}

\FloatBarrier
\subsubsection{Registro de usuarios}

Mediante el botón ``Registrarse'' se accede al registro de nuevos usuarios, donde se solicitarán una serie de datos de obligada cumplimentación. Una vez completo el registro, se recibe un pequeño regalo...\smiley:

\begin{figure} [!htb]
	\centering
	\resizebox{14.5cm}{!}{\includegraphics[width=\linewidth,height=\textheight,keepaspectratio]{Images/AnexoManual/02_registro_01_completo}}
	\caption{Registro de nuevos usuarios}
	\label{fig:registroUsuarios}
\end{figure}

A partir de este punto, se tiene acceso a operar con la aplicación.

\FloatBarrier
\subsubsection{Inicio de sesión}

Mediante el botón ``Iniciar Sesión'' los usuarios registrados pueden acceder a la aplicación, se solicitará el nombre de usuario y la contraseña seleccionados en la etapa de registro:

\begin{figure} [!htb]
	\centering
	\resizebox{12cm}{!}{\includegraphics[width=\linewidth,height=\textheight,keepaspectratio]{Images/AnexoManual/03_inicioSesion_01_completo}}
	\caption{Inicio de sesión}
	\label{fig:inicioSesion}
\end{figure}

\FloatBarrier
\subsubsection{Pantalla principal de la aplicación: vistas y opciones adicionales}

Una vez iniciada sesión, se accede a la pantalla principal de la aplicación: el mapa con la estaciones disponibles, que presentan cuatro códigos de colores posibles:

\begin{itemize}
	\item \textbf{\color{red} Rojo}. No quedan bicicletas disponibles en la estación.
	\item \textbf{\color{yellow} Amarillo}. Quedan menos del 50\% de bicicletas disponibles en la estación. En las estaciones con este nivel de disponibilidad la tarifa es el doble de lo habitual.
	\item \textbf{\color{green} Verde}. Quedan más del 50\% de bicicletas disponibles en la estación.
	\item \textbf{\color{blue} Azul}. El usuario tiene una reserva en esta estación.
\end{itemize}

\begin{figure} [!htb]
	\centering
	\resizebox{7cm}{!}{\includegraphics[width=\linewidth,height=\textheight,keepaspectratio]{Images/AnexoManual/04_pantallaPrincipal_01_completo}}
	\caption{Pantalla principal de la aplicación}
	\label{fig:pantallaPrincipal}
\end{figure}

\FloatBarrier
El estado concreto de una estación (bicis y anclajes disponibles y tarifa actual) se puede consultar mediante su selección:

\begin{figure} [!htb]
	\centering
	\resizebox{7cm}{!}{\includegraphics[width=\linewidth,height=\textheight,keepaspectratio]{Images/AnexoManual/04_pantallaPrincipal_02_detalleEstacion}}
	\caption{Detalle de una estación}
	\label{fig:detalleEstacion}
\end{figure}

Mediante los botones superiores se tiene acceso a otras vistas del parque de bicicletas y a opciones adicionales:

\begin{figure} [!htb]
	\centering
	\resizebox{7.5cm}{!}{\includegraphics[width=\linewidth,height=\textheight,keepaspectratio]{Images/AnexoManual/04_pantallaPrincipal_03_actionBar}}
	\caption{Diferentes vistas y opciones}
	\label{fig:vistasActionBar}
\end{figure}

El primero ofrece un listado desplegable con el detalle de cada estación. Aporta una vista diferente del conjunto de bicicletas que permite una búsqueda más sencilla y una visión más completa y detallada de cada una.

\begin{figure} [!htb]
	\centering
	\resizebox{9cm}{!}{\includegraphics[width=\linewidth,height=\textheight,keepaspectratio]{Images/AnexoManual/05_pantallaListado_01_completo}}
	\caption{Listado detallado del conjunto de estaciones}
	\label{fig:listadoEstaciones}
\end{figure}

El gráfico permite conocer el estado global del parque de estaciones, tanto para las bicicletas como para los anclajes, mostrando el porcentaje sobre el total de recursos disponibles, ocupados o reservados.

\begin{figure} [!htb]
	\centering
	\resizebox{13cm}{!}{\includegraphics[width=\linewidth,height=\textheight,keepaspectratio]{Images/AnexoManual/06_pantallaGrafico_01_completo}}
	\caption{Estado global del parque de estaciones}
	\label{fig:graficoEstadoGlobal}
\end{figure}

\FloatBarrier
Finalmente, el menú desplegable incluye las siguiente opciones adicionales:

\begin{figure} [!htb]
	\centering
	\resizebox{7cm}{!}{\includegraphics[width=\linewidth,height=\textheight,keepaspectratio]{Images/AnexoManual/04_pantallaPrincipal_04_actionBarDesplegable}}
	\caption{Menú desplegable y opciones adicionales}
	\label{fig:menuDesplegable}
\end{figure}

\begin{itemize}
	\item \textbf{Actualizar}. Actualiza el estado global del parque de estaciones. Si bien la aplicación se actualiza automáticamente con cada operación, esta opción habilita una actualización explícita. Señalar que las operaciones de actualización sobre las pantallas del listado de estaciones y el gráfico son análogas a esta.
	\item \textbf{Cuenta}. Acceso a la cuenta de usuario. Supone una opción adicional al acceso directo mostrado en la figura~\ref{fig:pantallaPrincipal}.
	\item \textbf{Configuración}. Acceso a diferentes configuraciones del sistema, más adelante en este manual se comenta esta opción en mayor detalle.
	\item \textbf{Cerrar sesión}. Mediante una confirmación adicional por parte del usuario, se cierra la sesión y se vuelve a la pantalla de inicio de sesión.
\end{itemize}

\subsubsection{Operar con bicicletas y anclajes}

Mediante la selección de una estación se muestra su estado en ese momento y se habilitan las diferentes opciones sobre la misma: coger una bicicleta, dejar una bicicleta y reservar una bicicleta y/o un anclaje.

\begin{figure} [!htb]
	\centering
	\resizebox{6cm}{!}{\includegraphics[width=\linewidth,height=\textheight,keepaspectratio]{Images/AnexoManual/07_pantallaOperaciones_01_opciones}}
	\caption{Opciones a la hora de operar con una estación}
	\label{fig:opcionesEstacion}
\end{figure}

De este modo, al coger una bici, se actualizará la barra de estado, además del saldo disponible:

\begin{figure} [!htb]
	\centering
	\resizebox{6cm}{!}{\includegraphics[width=\linewidth,height=\textheight,keepaspectratio]{Images/AnexoManual/07_pantallaOperaciones_02_biciCogida}}
	\caption{Actualización de la barra de estado al coger una bici}
	\label{fig:barraEstadoBiciCogida}
\end{figure}

\FloatBarrier
Barra de estado que volverá a su estado inicial cuando se deje la bicicleta.

Por su parte, al seleccionar la reserva de recursos, se habilitará un diálogo que permitirá seleccionar el elemento que queramos reservar (sólo aparecerán habilitadas para su selección las opciones posibles para el usuario dadas las restricciones consideradas en esta operación, ver la especificación del requisito \emph{RF08 - Reservar} de la sección~\ref{sec:secERS} para más detalles). Una vez completada la reserva, la estación afectada quedará marcada en color azul, para una mejor identificación.

\begin{figure} [!htb]
	\centering
	\resizebox{12cm}{!}{\includegraphics[width=\linewidth,height=\textheight,keepaspectratio]{Images/AnexoManual/07_pantallaOperaciones_03_reserva}}
	\caption{Reserva de bicicleta y anclajes}
	\label{fig:reservaBicisAnclajes}
\end{figure}

\subsubsection{Cuenta de usuario}

Una vez se haya accedido a la cuenta de usuario mediante alguna de las opciones mencionada, se habilitan las siguientes opciones:

\begin{itemize}
	\item \textbf{Consulta y cancelación de reservas}. La cancelación requiere de una confirmación adicional por parte del usuario.
	\item \textbf{Consulta e ingreso de saldo}.
	\item \textbf{Consulta, edición y borrado del perfil de usuario}. La edición de perfil supone una actividad similar al registro de usuario, mientras que el borrado de la cuenta requiere de una confirmación adicional por parte del usuario.
\end{itemize}

\begin{figure} [!htb]
	\centering
	\resizebox{12cm}{!}{\includegraphics[width=\linewidth,height=\textheight,keepaspectratio]{Images/AnexoManual/08_pantallaCuenta_01_completa}}
	\caption{Cuenta de usuario}
	\label{fig:cuentaUsuario}
\end{figure}

\subsubsection{Configuración}

Mediante el acceso a esta pantalla a través del menú de opciones adicionales comentado antes, se habilitan las siguientes opciones:

\begin{itemize}
	\item \textbf{Cambiar los datos de conexión con el servidor}. Análogo al procedimiento comentado al inicio de este manual.
	\item \textbf{Habilitar el modo ``superusuario''}. Mediante la activación de este modo no se comprueban restricciones de usuario a la hora de coger o dejar bicis (se siguen comprobando las restricciones habituales para la reserva de recursos y las relativas a las estaciones, es decir, si una estación no tiene bicis, no se pueden coger). El objetivo de esta opción es el desarrollo de pruebas sobre el software de un modo más ágil.
\end{itemize}

\begin{figure} [!htb]
	\centering
	\resizebox{6cm}{!}{\includegraphics[width=\linewidth,height=\textheight,keepaspectratio]{Images/AnexoManual/09_Configuraciones_01_completa}}
	\caption{Configuraciones posibles}
	\label{fig:configuracionesPosibles}
\end{figure}

En el caso de activar el modo ``superusuario'', la barra de estado informará de ello:

\begin{figure} [!htb]
	\centering
	\resizebox{6cm}{!}{\includegraphics[width=\linewidth,height=\textheight,keepaspectratio]{Images/AnexoManual/09_Configuraciones_02_super}}
	\caption{``Superusuario'' activado}
	\label{fig:superusuarioActivado}
\end{figure}

\subsubsection{FAQs}

Se recoge un conjunto de incidencias comunes en el uso básico de la aplicación con su posible resolución, partiendo del supuesto que los servidores de aplicación y base de datos están correctamente configurados y conectados entre sí.

\begin{itemize}
	
	\item \textit{Al registrar un usuario, la aplicación no me deja porque el nombre de usuario o dirección de correo no están disponibles}. Hay otro usuario registrado con ese nombre de usuario o dirección de correo, debes introducir uno diferente.
	
	\item \textit{Al iniciar sesión, la aplicación no me lo permite por no encontrar el usuario o por contraseña incorrecta}. Asegúrate de que el nombre de usuario y contraseña introducidos para el inicio de sesión coinciden con los registrados.
	
	\item \textit{A pesar de estar correctamente conectado, no puedo coger una bici}. Es posible que estés incurriendo en alguna de las restricciones impuestas para esta operación, ver la especificación del requisito \emph{RF06 - Coger bicicleta} de la sección~\ref{sec:secERS} para más detalles.
	
	\item \textit{A pesar de estar correctamente conectado, no puedo dejar una bici}. Es posible que estés incurriendo en alguna de las restricciones impuestas para esta operación, ver la especificación del requisito \emph{RF07 - Dejar bicicleta} de la sección~\ref{sec:secERS} para más detalles.
	
	\item \textit{A pesar de estar correctamente conectado, no puedo reservar una bicicleta y/o anclaje}. Es posible que estés incurriendo en alguna de las restricciones impuestas para esta operación, ver la especificación del requisito \emph{RF08 - Reservar} de la sección~\ref{sec:secERS} para más detalles.
	
	\item \textit{Tenía una reserva que no he cancelado y ha desaparecido a pesar de que no he hecho uso de ella}. Posiblemente habrán pasado más de 30 minutos desde la reserva. Este tiempo es el máximo establecido para hacer uso de la reserva, una vez superado se cancela automáticamente.
	
	\item \textit{No puedo borrar mi perfil}. Es posible que tengas una bicicleta cogida, debes dejarla en una estación antes de borrar el perfil.
	
	\item \textit{He borrado mi cuenta, ¿puedo recuperarla?}. No, el borrado de cuenta, una vez confirmado por el usuario, es definitivo, con lo que el acceso a la aplicación sólo se puede realizar registrando uno nuevo.
	
\end{itemize}





