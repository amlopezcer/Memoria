\chapter{Conclusiones}

NO ME GUSTA MUCHO ESTO, NO SÉ SI LAS CONCLUSIONES SE ENFOCAN TANTO A LO PERSONAL

El presente proyecto ha supuesto la elaboración de una aplicación Android completa comunicada de manera remota con su servidor y base de datos mediante el estándar \textsc{REST} de los servicios web.

Dadas las diferentes tecnologías y capas de datos utilizadas, el conocimiento técnico adquirido ha sido amplio, permitiéndome conocer, comparar y seleccionar diferentes soluciones de bases de datos, servidores, transmisión y serialización de datos\dots dando lugar a una visión final del desarrollo software para dispositivos Android bastante sólida y completa.

Desde el punto de vista de las capacidades personales, cabría destacar la mejora y asentamiento de las capacidades propias de organización, análisis y síntesis, que me han permitido llevar a buen puerto la herramienta desarrollada.

Más allá de los apartados personales, se espera que la herramienta pueda ser utilizada como base de estudio para la mejora de los servicios de gestión de parques públicos de bicicletas\dots SEGUIR AÑADIENDO COSAS\dots


\section{Líneas futuras}

La aplicación supone una base tecnológica para el desarrollo y mejora de los servicios de gestión de parques públicos de bicicletas.

De manera concreta, se consideran las siguientes líneas futuras de desarrollo y estudio:

\begin{itemize}  
	\item Desarrollo de la adaptación de precios a la hora de coger, dejar o reservar bicicletas, dependiendo de la disponibilidad de cada estación.
	\item Desarrollo de un sistema de alertas automático que avise al usuario de la disponibilidad de bicicletas o anclajes en una estación elegida por él.
	\item Desarrollo de versiones sobre diferentes plataformas (web, tablet, etc.), con el objetivo de llegar a un público más amplio.
\end{itemize}