\chapter{Conclusiones}

El presente proyecto ha supuesto el desarrollo de una aplicación Android comunicada de manera remota con su servidor y base de datos para la gestión de parques públicos de bicicletas.

Sobre el sistema planteado inicialmente, se considera que se han cumplido los objetivos planteados en el capítulo~\ref{ch:objetivos}. Además del básico referente a la construcción del sistema con sus funcionalidades principales, se ha hecho un estudio de las tecnologías involucradas para su selección fundamentada; el entendimiento de los estándares de diseño e implementación, en sus diferentes ámbitos, ha sido concienzudo; así como el desarrollo de una estrategia de pruebas adecuada, comenzando por las unidades más pequeñas de los elementos software y llegando al agregado que suponen los requisitos funcionales.

Dada la magnitud del proyecto, con una arquitectura Cliente/Servidor de tres capas, las tecnologías utilizadas han sido numerosas, desde las involucradas en las diferentes capas de datos (cliente, servidor y base de datos), hasta aquellas requeridas para la comunicación entre ellas. Ello ha conllevado un aprendizaje y comprensión adecuada de lo que supone el despliegue de una aplicación completa.

El desarrollo sobre un sistema Android ha sido satisfactorio. Superadas las dificultades iniciales mediante el estudio y entendimiento adecuados de las particularidades del sistema, el desarrollo es relativamente ágil gracias a la cantidad de documentación oficial publicada y a la comunidad de desarrolladores presentes en foros especializados. Asimismo, gracias a su predominancia en el mercado de los dispositivos móviles por un lado y las soluciones ideadas para soportar diversos dispositivos (lenguajes, tamaños de pantalla, versiones del sistema, etc.) por otro, las aplicaciones desarrolladas tienen un potencial de usuarios mayor que otras plataformas.

Más allá de los aspectos técnicos, están aquellos referentes a la documentación y a la Ingeniería del Software. Mediante el apoyo de una adecuada bibliografía, se considera que el sistema ha quedado adecuadamente descrito y probado en sus diferentes etapas.

Desde un punto de vista personal, las principal dificultad ha sido la falta de tiempo. Compatibilizar la realización del proyecto con un empleo especialmente absorbente desemboca en fines de semana de dedicación intensa para poder tener un grado de avance aceptable, seguidos de paradas de días hasta el fin de semana siguiente, reduciendo la continuidad y la productividad. Sin embargo, no deja de ser una dificultad que se supera mediante una adecuada mentalización y organización. Adicionalmente, quedan las dificultades habituales como son, por ejemplo, el desconocimiento de muchas de las tecnologías requeridas. Si bien aspectos como este terminan en un aporte de conocimientos técnicos y organizativos intenso y enriquecedor.

\section{Líneas futuras}

La aplicación supone una base tecnológica para el desarrollo y mejora de los servicios de gestión de parques públicos de bicicletas. Se consideran las siguientes posibles líneas futuras de desarrollo y estudio:

\begin{itemize}  
	\item Desarrollo de la adaptación de precios a la hora de coger, dejar o reservar bicicletas, dependiendo de la disponibilidad de cada estación. Actualmente la aplicación duplica el precio para coger una bicicleta si la disponibilidad de la estación cae por debajo del 50\%.
	\item Desarrollo de un sistema de alertas o notificaciones automático que avise al usuario de la disponibilidad de bicicletas o anclajes en una estación elegida por él. De este modo el usuario podría seleccionar una serie de estaciones ``favoritas'' sobre las que periódicamente se informe de la disponibilidad o, en caso de que haya querido reservar y no hubiese sido posible, alerte cuando haya una bici preparada.
	\item Inclusión de un mayor volumen de estados para las bicicletas. En la aplicación actual, una bici puede estar disponible, reservada o cogida, se podría incluir un nuevo estado que hiciese referencia a, por ejemplo, las bicicletas averiadas, de modo que un usuario que identificase una avería en una bici o estación informase mediante una breve descripción a través de la aplicación.
	\item Desarrollo de versiones sobre diferentes plataformas. Actualmente la aplicación se encuentra desarrollada únicamente para dispositivos Android, se podría realizar una adaptación de IU para tablets, desarrollo para otros sistema operativos como iOS o plataformas como una versión web.
	\item Relacionado con el punto anterior, la aplicación Android se debería mantener actualizada y adaptada las nuevas versiones o funcionalidades que se vayan publicando.
	\item Extender la batería de pruebas para llegar a cubrir las \emph{pruebas de sistema} fundamentales, extensión que conllevaría una adaptación del sistema diseñado de cara a superarlas y dando lugar a un entorno más consistente y profesional.
\end{itemize}