
\chapter{Resumen}


El presente proyecto ha supuesto el desarrollo de una aplicación Android completa para la gestión de parques de bicicletas públicos, estableciendo una infraestructura tecnológica base para desarrollos e investigaciones futuras. 

Aplicaciones como la presentada cobran gran importancia en la actualidad dados dos factores básicos: por un lado, la utilización, cada vez mayor, de la bicicleta pública como alternativa a los medios de transporte tradicionales; por otro lado, la dominancia de los \emph{smartphones} como herramientas de gestión de tareas diarias, como podría ser la reserva de un bicicleta. Así, se hace necesario contar con aplicaciones y sistemas eficientes, eficaces y fácilmente utilizables para la gestión de dichos parque públicos.

De este modo, la herramienta presenta un mapa actualizado de las estaciones disponibles sobre el que poder operar, así como pantallas adicionales para facilitar la interacción del usuario con la aplicación a la hora de gestionar su perfil o conocer el estado del parque de bicicletas.

Desde un punto de vista técnico, la aplicación sigue una arquitectura cliente-servidor de tres capas, siendo Glassfish y SQL Server el soporte tecnológico seleccionado para el servidor y la base de datos, respectivamente. De este modo, la comunicación entre el cliente y el servidor se realiza mediante servicios web y el protocolo HTTP utilizado por el estándar REST. 

El desarrollo ha seguido un modelo de proceso incremental bajo un paradigma orientado a objetos mediante el lenguaje de programación Java, tratando de tener en todo momento en mente las buenas prácticas establecidas por la Ingeniería del Software, las \emph{reglas de oro} para el diseño de interfaces de usuario, recomendaciones de diseño e implementación Android, etc. Dado el modelo señalado, el diseño e implementación se ha dividido en una serie de incrementos jerarquizados por importancia y desarrollados de manera individual para, finalmente, quedar integrados en el producto final.

La aplicación ha sido probada mediante pruebas planificadas de unidad, de integración y de validación para tratar de lograr una elevada cobertura de errores y calidad final.

Finalmente, señalar que como se ha indicado al principio de este resumen, la herramienta presentada supone una infraestructura base para investigaciones futuras que conduzcan a modelos de gestión más eficientes y con mayor capacidad de atracción de usuarios que los utilizados actualmente.




