
\chapter{Resumen}


El presente proyecto ha supuesto la elaboración de una aplicación Android distribuida para la gestión de parques de bicicletas, estableciendo una infraestructura tecnológica base para desarrollos e investigaciones futuras. 

La aplicación permite simular las acciones de coger, dejar y reservar bicicletas y anclajes entre un conjunto de estaciones. Se introduce, además, un primer nivel de adaptación dinámica de precios acorde a la disponibilidad de cada estación, de manera que coger una bicicleta de una estación con baja disponibilidad supone una tarifa mayor que en otro puesto.

La herramienta presenta un mapa actualizado de las estaciones disponibles sobre el que poder operar, así como pantallas adicionales para facilitar la interacción del usuario con la aplicación a la hora de gestionar su perfil o conocer el estado del parque de bicicletas.

Desde un punto de vista técnico, la aplicación sigue una arquitectura Cliente/Servidor de tres capas, estructurándose la comunicación mediante el estándar REST de los servicios web. 

El desarrollo ha seguido un modelo de proceso incremental bajo un paradigma orientado a objetos, tratando de tener en todo momento en mente las buenas prácticas establecidas por la Ingeniería del Software, las \emph{Reglas de Oro} para el diseño de interfaces de usuario, recomendaciones de diseño e implementación Android, etc. Dado el modelo señalado, el proceso se ha dividido en una serie de incrementos jerarquizados por importancia y desarrollados de manera individual para, finalmente, quedar integrados en el producto final.

La aplicación ha sido probada mediante pruebas planificadas de unidad, de integración y de validación para tratar de lograr una elevada cobertura de errores y calidad final.

Se espera, como se ha indicado, que la herramienta presentada suponga una infraestructura base consistente para investigaciones futuras que conduzcan a modelos de gestión más eficientes y eficaces.




